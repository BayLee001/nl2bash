\section{Background and Related Work}
The idea of natural language based computer interface was born even before the UNIX sytem was released~\cite{sammet1966use,Ballard:1979:PNL:800177.810072}. The central idea was to build compilers that compile natural language into lower-level instructions. However, NLP techniques were preliminary back then, and those systems, relying on manually engineered templates and grammars, had failed to handle the rich semantics and ambiguity in natural languages. 
% Besides, some researchers pointed out that making machines to understand our native tongue is a fruitless trial for it lacks mathematical support and programming languages remains simple in the formal system. In short, it was easier for people to learn how to program than for machines to learn natural language. 

While it may be unrealistic (and also fruitless~\cite{Dijkstra:1978:FNL:647639.760596}) to simply adopt natural language as a programming language, recent work have shown promise on synthesizing high-level code snippets from natural language descriptions~\cite{gulwani2010dimensions,DBLP:journals/corr/DesaiGHJKMRR15}.  
% and can be furthur improved through interaction with users~\cite{DBLP:conf/sigmod/GulwaniM14,DBLP:journals/pvldb/LiJ14}.
Despite the initial success of natural language interface to database~\cite{Popescu:2003:TTN:604045.604070,DBLP:journals/pvldb/LiJ14} and commercial applications such as the Microsoft Excel~\cite{DBLP:conf/mobisys/LeGS13,DBLP:conf/acl/QuirkMG15}, previous work on shell programming from natural language have shown that the task is extremely difficult~\cite{bashsynthesis,cozzie2011macho,cozzie2012macho,Pedersen-Report}. 
% In contrast to the previously introduced programming by natural language techniques, 
Macho~\cite{cozzie2011macho} is a system that can extract specifications from natural language text in the Linux man-pages\footnote{https://www.kernel.org/doc/man-pages/}, and generates candidate programs based on that. However, their approach only worked on generating the head commands.~\cite{cozzie2012macho} combines the natural language specifications with an example given by the user, and shows that the combination effectively prunes the search space, but the results were still limited to commands of length 2 or smaller.~\cite{bashsynthesis} attempted to synthesis file system operation commands using StackOverflow question-answer pairs, and have succeed in generating commands of length 5 or smaller. However, due to the lack of high-quality training data, big challenge still remains for the system to be practically useful.

