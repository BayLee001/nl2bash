%!TEX root=writeup.tex
\section{Evaluation}

Our key research question is:
%
\begin{quote}
    Can the tool described thus far improve the speed with which programmers
    accomplish tasks with the command line?
\end{quote}

\subsection{Methodology}
\label{subsec:methodology}

To answer this question we plan to conduct a small pilot user study. Programmers
will be asked to accomplish various tasks at the command line. They will be
divided into two groups: one group will be allowed to use our tool, and the
other will be allowed to use the internet. Both groups will be allowed the use
of manpages and will be allowed to experiment by running their commands.

We hypothesize that our tool will improve the speed with which programmers
devise solutions without impeding the correctness of those solutions.

\paragraph{Task} The programmers will each be asked to perform a small number of
tasks using the command line. The tasks are shown in \autoref{fig:tasks} and
will be drawn from our collected data. Importantly, these tasks will be set
aside before training the tool; they will not be examples the tool has seen
before.

There are several confounding factors to account for: some tasks may be easier
than others, some participants may be more experienced than others, and the
tasks may become easier for the participants as they progres. To control for
these, the tasks have been randomly split into two blocks $T_1$ and $T_2$.
Each participant will each block under one of two setups:
\begin{itemize}
    \item $S_1$ --- access to local manpages and the internet
    \item $S_2$ --- access to local manpages and our tool
\end{itemize}

There are four possible combinations of (block, setup) assignments and
orderings; participants will be randomly assigned to one of these four groups:
\begin{itemize}
    \item $T_1 S_1$, $T_2 S_2$ --- task 1 under setup 1 followed by task 2 under setup 2
    \item $T_1 S_2$, $T_2 S_1$ --- task 1 under setup 2 followed by task 2 under setup 1
    \item $T_2 S_1$, $T_1 S_2$ --- task 2 under setup 1 followed by task 1 under setup 2
    \item $T_2 S_2$, $T_1 S_1$ --- task 2 under setup 2 followed by task 1 under setup 1
\end{itemize}

While the tasks all have one-line solutions, those solutions may involve exotic
commands or flags. The programmers will be allowed to produce multi-line scripts
as solutions, provided that those scripts correctly automate the task. We will
not impose any specific time limit on the participants.

\begin{figure}[ht]
    \begin{subfigure}[b]{0.48\textwidth}
        \begin{framed}
        \begin{itemize}\itemsep-1pt
            \item In the current directory, recursively find all files with ``conf''
                in the filename
            \item Recursively remove all empty sub-directories from a directory tree
            \item Find files that are not executable
            \item Find the 100 biggest files on your system
            \item Make a tar archive of a folder, excluding .png files
        \end{itemize}
        \end{framed}
        \caption{Task set $T_1$}
    \end{subfigure}
    \begin{subfigure}[b]{0.48\textwidth}
        \begin{framed}
        \begin{itemize}\itemsep-1pt
            \item ...
            \item ...
            \item ...
            \item ...
            \item ...
        \end{itemize}
        \end{framed}
        \caption{Task set $T_2$}
    \end{subfigure}
    \caption{Tasks for the user study, randomly divided into two groups.}
    \vspace{-10pt}
    \label{fig:tasks}
\end{figure}

\paragraph{Participants} The participants in the study will be undergraduate and
graduate students. They will be proficient command-line users, but not
necessarily experts. We expect to perform the study with at least five
participants in each group.

\paragraph{Measurements} We will primarily measure two things:
\begin{enumerate}\itemsep-1pt
    \item How long do participants take to perform each task?
    \item Are the participants' solutions correct?
\end{enumerate}
We will measure the participants' speed by having them specifically indicate
when they have completed each task. We will measure correctness by having them
save the output from each task; we will compare that output against the correct
solution.

\paragraph{Internal Measures} Additionally, we plan to quantify several internal
measures about the tool:
\begin{enumerate}\itemsep-1pt
    \item Performance: how long does training take to produce a model? How
        long does the tool take to answer a query?
    \item For a large number of hand-picked inputs, how often does the
        tool produce a correct answer?
    \item How often is the output close to correct, requiring only small
        adjustment? (``Small adjustment'' is defined as removing flags or
        changing arguments. Adding an argument, changing a command, or
        adding a pipe operator are not small adjustments.)
\end{enumerate}

\subsection{Results}

The current prototype differs somewhat from the vision put forward in
\autoref{sec:framework}. While we can automatically parse many manpages, this is
not connected to the tool---instead, the output grammar is defined by a limited
set of hand-written rules. Additionally, the scoring function is based on simple
co-occurrence of English words and bash tokens.

\autoref{tbl:stats} shows a few simple statistics about the training data and
output.

Training the entire model---which is very simple for co-occurrence---takes only
44 seconds.

\autoref{tbl:manual-inspection} shows the top-ranked result for a few very
simple inputs. Performance numbers are relatively static since performance
depends largely on the length of the input string and the complexity of the
output grammer. Since the grammar is hand-coded right now, the tool does not
have many options to consider.

Examples 1 and 2 are mostly correct. The inclusion of the \texttt{-v} flag is
optional and the inclusion of the \texttt{-z} flag might be a mistake if the
target archive was not compressed with GZip.

Example 3 is also mostly correct. The question did not specify anything about
filtering files by name, but \texttt{find} with the \texttt{-delete} flag is
one way to accomplish this task. A solution using \texttt{rm} appears third in
the rankings, after an erroneous suggestion involving \texttt{xargs}.

Example 4 is correct. It is unclear whether \texttt{-e} and \texttt{-i} flags
are the right thing to do, but a recursive \texttt{grep} is a very good way to
answer this query.

Example 5 is incorrect. The tool did not assign a high enough weight to the word
``remote,'' and so it missed the correct answer which uses \texttt{scp}. The
correct answer appears third in the rankings, below another incorrect answer
involving \texttt{find}.

These preliminary results suggest a better scoring metric is needed. The
co-occurrence model gives too much weight to common commands such as
\texttt{find} and \texttt{cp} for specialized queries, and is too likely to
return extraneous flags. Additionally, we need a way to identify words or
phrases that appear in the question and should appear in the answer, such as
``ERROR'' from example 4.


\begin{table}
    \begin{subfigure}[b]{0.48\textwidth}
        \begin{center}
        \begin{tabular}[t]{lr}
            \textbf{Stat} & \textbf{Value} \\
            \hline
            \# of questions                & 42111 \\
            \# of (question, answer) pairs & 75518 \\
            \# of parseable pairs          & 29880 \\
        \end{tabular}
        \end{center}
        \caption{Training data statistics}
    \end{subfigure}
    \begin{subfigure}[b]{0.48\textwidth}
        \begin{center}
        \begin{tabular}[t]{lr}
            \textbf{Stat} & \textbf{Value} \\
            \hline
            Training time & 44 s \\
        \end{tabular}
        \end{center}
        \caption{Learning statistics}
    \end{subfigure}
    \caption{Various internal statistics about the tool and the training data.}
    \label{tbl:stats}
\end{table}

\begin{table}
    \begin{center}
    \begin{tabular}{rlrl}
        \# & \textbf{Input} & \textbf{Time (s)} & \textbf{Output} \\
        \hline
        1  & ``extract a tar archive'' & 0.23 & \texttt{tar -x -v -z -f FILE.TAR} \\
        2  & ``create a tar archive''  & 0.22 & \texttt{tar -c -v -z -f FILE.TAR} \\
        3  & ``recursively delete files from a directory'' & 0.24 & \texttt{find PATH -name PATTERN -delete} \\
        4  & ``locate files containing the string ERROR''  & 0.24 & \texttt{grep -e -i -r REGEX PATH} \\
        5  & ``copy a directory to a remote machine'' & 0.23 & \texttt{cp -R PATH DEST} \\
    \end{tabular}
    \end{center}
    \caption{Performance and manual inspection data. The ``Input'' column lists
        natural-language inputs to the tool. The ``Time (s)'' column lists times
        (in seconds) to produce an answer. The ``Output'' column shows the
        tool's top-ranked result.}
    \label{tbl:manual-inspection}
\end{table}

\paragraph{Sequence-to-sequence Baseline Results} Another baseline model we have 
set up for this problem is the classical natural language machine translation 
model (IBM Model 4\footnote{\url{https://en.wikipedia.org/wiki/IBM_alignment_models\#Model_4}}). 
This model treats the command line programs as a foreign language, and translates 
the English commands into it using probabilistic sequence modeling. We trained a 
naive translation model (without fine tuning the parameters) using the dataset shown
in table~\ref{tbl:stats}. The results are shown in table~\ref{tbl:manual-inspection-ibm4}.
Compared to the rule-based model introduced above, we observed that it is very common for 
the sequence-based model to generate ungrammatical commands, since the output decoding 
is sequential and does not obey any constraints of the grammatical structure. Yet surprisingly, 
it almost generates command no.2 completely correct, and correctly suggested the head 
command of command no.5. Especially, command no.5 is easy to trick a system into suggesting 
``\texttt{cp}''. However, since IBM Model 4 handles multi-to-multi word mappings, it is 
able to capture the relation between ``copy ... to a remote machine'' and ``\texttt{rsync}''.
Based on the comparison above, we think a good model for this problem should combine the
strength of both rule-based modeling and probabilistic multi-to-multi mapping learnt
from data.

\begin{table}
    \begin{center}
    \begin{tabular}{rlrl}
        \# & \textbf{Input} & \textbf{Time (s)} &{IBM Model 4} \\
        \hline
        1  & ``extract a tar archive'' & $>$ 1 & \texttt{grep -o -P tar} \\
        2  & ``create a tar archive''  & $>$ 1 & \texttt{tar -c -f test/foo/test.tar} \\
        3  & ``recursively delete files from a directory'' & $>$ 1 & \texttt{find [} \\
        4  & ``locate files containing the string ERROR''  & $>$ 1 & \texttt{-or -name \$0 ERROR} \\
        5  & ``copy a directory to a remote machine'' & $>$ 1 & \texttt{rsync -a find `find} \\
    \end{tabular}
    \end{center}
    \caption{Sequence-to-sequence translation based results by IBM Model 4. The ``Input'' column lists
        natural-language inputs to the tool. The ``Time (s)'' column lists times
        (in seconds) to produce an answer. The ``Output'' column shows the
        tool's top-ranked result.}
    \label{tbl:manual-inspection-ibm4}
\end{table}

\begin{table}
    \begin{center}
    \begin{tabular}{lrr}
        \textbf{Task/Setup} & \textbf{Without (s)} & \textbf{With (s)} \\
        \hline
        $T_1 S_1$, $T_2 S_2$ & - & - \\
        $T_1 S_2$, $T_2 S_1$ & - & - \\
        $T_2 S_1$, $T_1 S_2$ & - & - \\
        $T_2 S_2$, $T_1 S_1$ & - & - \\
    \end{tabular}
    \end{center}
    \caption{User study results. The ``Task/Setup'' column corresponds to the
        groups outlined in \autoref{subsec:methodology}. The ``Without (s)''
        column shows the average time taken by participants on the task with
        setup 1 (manpages+internet). The ``With (s)'' column shows the average
        time taken by participants on the task with setup 2 (manpages+tool).}
\end{table}
