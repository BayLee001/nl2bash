%!TEX root=writeup.tex
\section{Introduction}

What is the problem?
Why is it important?
What makes it hard?
How will we solve it?

Command line interface (CLI) provides the convenience for user to perform a set of rich and meaning operations including file system manipulation, remote control and system configuration etc. Due to the importance of CLI programming in modern software development, whether one can proficiently programming in CLI is commonly used as a criteria for companies to hire programmers. 

Comparing to graphical user interface (GUI), CLI provides more flexible control of the operating system, with better performance, and can cover a larger set of different works. However, the richness nature of instructions in CLI requires a higher degree of memorization and familiarity, and as a result, new users often find it harder to grasp than GUI. As early as 1990's, researchers started the explorations different ways to make CLI programming easier, and one promising solution is natural language programming~\cite{Pederson-Report,Manaris:1994:DNL:198125.198137,ZOLTANFORD1991527}: such systems asks users to input natural language descriptions of the task, and it translates these descriptions into executable commands to complete the task. However, though natural language programming techniques has been studied cross disciplinary for a long time and show its success in several domains, e.g. database query~\cite{DBLP:journals/pvldb/LiJ14, DBLP:conf/sigmod/GulwaniM14}, text editing~\cite{DBLP:journals/corr/DesaiGHJKMRR15} and smart phone scripting~\cite{DBLP:conf/mobisys/LeGS13}. The solution for CLI interface is nevertheless satisfying. Key challenges in addressing this problem are listed below.
\begin{itemize}
\item \textbf{Ambiguity.} Natural language descriptions are ambiguous in its nature and thus clarify their meaning is difficult. Though several techniques including interactive disambiguation~\cite{DBLP:journals/pvldb/LiJ14} and keyword-based translation~\cite{DBLP:conf/sigmod/GulwaniM14} are proposed, ill-formed sentences remained impossible to handle as these techniques requires a well formed parse tree before further processing. Unfortunately, ill-formed descriptions is a common case for command line questions according to investigation results from online forums including StackOverflow\footnote{http://stackoverflow.com/questions/2193584/copy-folder-recursively-excluding-some-folders}, UnixExchange\footnote{http://unix.stackexchange.com/questions/tagged/bash} and BashOneLiner\footnote{http://www.bashoneliners.com/oneliners/oneliner/popular/}.
\item \textbf{Richness in basic operations.} Unlike other questions like database queries, which are complex in their program structure but simple in atomic operations, command line programs are often simple in structure but complex in basic operations. As the set of basic commands are large and keeps growing, system designers are unable to statically encoding all basic operations in the system, and the natural language interface for CLI requires some mechanism to automatically learn basic operations from corpus to keep the system up-to-date.
\end{itemize}

\todo{Write more later}