%!TEX root=writeup.tex
\section{Experimental Evaluation}

Our key research question is:
%
\begin{quote}
    Can the tool described thus far improve the speed with which programmers
    accomplish tasks with the command line?
\end{quote}
%
To answer this question we plan to conduct a small pilot user study. Programmers
will be asked to accomplish various tasks at the command line. They will be
divided into two groups: one group will be allowed to use our tool, and the
other will be allowed to use the internet. Both groups will be allowed the use
of manpages and will be allowed to experiment by running their commands.

We hypothesize that our tool will improve the speed with which programmers
devise solutions without impeding the correctness of those solutions.

\paragraph{Task} The programmers will each be asked to perform a small number of
tasks using the command line. The tasks will be drawn from our collected data.
A straightforward one-line solution exists for each task, although that solution
may use exotic commands or flags. The tasks will not be inputs that the tool has
been trained on. Some example tasks are listed in \autoref{fig:sample-tasks}. We
will not impose any specific time limit on the participants.

\begin{figure}[ht]
    \begin{framed}
    \begin{itemize}\itemsep-1pt
        \item In the current directory, recursively find all files with ``conf''
            in the filename
        \item Recursively remove all empty sub-directories from a directory tree
        \item Find files that are not executable
        \item Find the 100 biggest files on your system
        \item Make a tar archive of a folder, excluding .png files
    \end{itemize}
    \end{framed}
    \caption{Sample tasks for the proposed user study.}
    \vspace{-10pt}
    \label{fig:sample-tasks}
\end{figure}

\paragraph{Participants} The participants in the study will be undergraduate and
graduate students. They will be proficient command-line users, but not
necessarily experts. We expect to perform the study with five participants in
each group.

\paragraph{Measurements} We will primarily measure two things:
\begin{enumerate}\itemsep-1pt
    \item How long do participants take to perform each task?
    \item Are the participants' solutions correct?
\end{enumerate}
We will measure the participants' speed by having them specifically indicate
when they have completed each task. We will measure correctness by having them
save the output from each task; we will compare that output against the correct
solution.
